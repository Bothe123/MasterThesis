\chapter{Conclusions}

% question was how eta in generalized model changes things
The main question in this thesis was, how the introduction of a new acceleration parameter $\eta$ into the widely used L\'evy walk model changes its properties, and in particular what its effect on the recently found divergence in the \gls*{msd} of the L\'evy walk are. \\
To this end the asymptotic behavior of the \gls*{msd} was calculated in one dimension, both for the ordinary and for the aged walk, which can be found in 
\cite{bothe}. 
We showed that the \gls*{msd} is finite when the condition $\gamma > 2(\nu-\eta)$ is satisfied, not just in the ordinary case, where this was already known\cite{radons2018}, 
but for the aged case as well. This means that the introduction of $\eta$ is indeed suitable to avoid the divergence of the original model. \\
Furthermore we found that $\eta$ does not affect the asymptotic time dependence of the \gls*{msd}, meaning that for the ordinary walk, all diffusion regimes can be recovered from the original model for suitable values of $\eta$, including the superballistic Richardson regime. \\
Additionally it was discovered that in the limit of long aging times neither subdiffusion nor superballistic diffusion can be found in the generalized model.

A similar calculation attempted to find the asymptotic behavior of the \gls*{PDF}, but this could unfortunately not be done analytically for general parameter values. It was however possible to extract the correct time dependence of the probability density at the origin and to derive the analytical solution in the domain of applicability of the central limit theorem. For the other cases a numerical evaluation of the expression resulted in an approximation of the \gls*{PDF} that was not able to capture the rich behavior revealed in a simulation of the process. 

For this simulation CUDA parallel computing was used to validate the findings for the \gls*{msd}, and to create histograms for both the aged and the ordinary case. Here we find significant differences in the \gls*{PDF} for different values of the parameters $\gamma$ and $\nu$. Interestingly the new parameter $\eta$ also has a major influence on the \gls*{PDF}, as it governs the formation of delta peaks as well as the existence of a cutoff for the \gls*{PDF}, which explains $\eta$'s role in the convergence condition discussed above. The results can mostly be explained heuristically by analyzing the influence of long steps on the process. \\
We also find significant aging effects for the \gls*{PDF}, which can be explained by a similar analysis.

A goal for future work is to find an analytical expression that can describe the features of the PDF quantitatively. One approach could be the incorporation of higher order terms into the asymptotic expansion in this work, which can be done straightforwardly. It is however not immediately clear if this would capture the effect of $\eta$ on the \gls*{PDF}, so another approach might be more suitable.\\
It should also be noted that this thesis focuses very much on the theoretical aspects of the model, so it should be studied how the different values of $\eta$ could be realized in experimental settings. A natural candidate would be the case $\eta=\nu=2$, corresponding to a process with constant acceleration, which has already been investigated for a similar model in \cite{burioni2013rare, burioni2014scaling}.

To summarize, we show that the generalized L\'evy walk does indeed allow for the recovery of the diffusion regimes that were divergent in the original model. Furthermore we find that he introduction of $\eta$ has a significant impact on the \gls*{PDF}, which should be investigated further.

