\chapter{Conclusions}

% question was how eta in generalized model changes things
The question I set out to answer in this thesis is how the introduction of a new acceleration parameter $\eta$ into the widely used L\'evy walk model would change its properties and in particular what its effect on the recently found divergence in the \gls{msd} of the model would be. 

To this end the asymptotic behavior of the \gls{msd} was calculated, both in the ordinary and in the aged walk, which can be found in \cite{bothe}. Both cases are only finite when the condition $\gamma > 2(\nu-\eta)$ is satisfied, showing how $\eta$ governs the convergence. Other than this the new parameter does not affect the asymptotic time dependence, meaning that for the ordinary walk all regimes of the original model can be recovered for suitable values of $\eta$, including the superballistic Richardson regime. 
{\color{blue}
Furthermore it was shown that in the limit of long aging times neither subdiffusion and superballistic diffusion can be found in the model, which was previously unknown.
}

A similar calculation was performed to find the asymptotic behavior of the \gls{PDF}, which unfortunately could not be calculated for general parameters analytically. It was however possible to extract the correct time dependence of the probability density at the origin and the domain of influence of the central limit theorem. For the other cases a numerical evaluation of the expression resulted in a \gls{PDF} that was unfortunately not able to capture the rich behavior revealed in a simulation of the process. \\
For this CUDA parallel computing was utilized to validate the findings for the \gls{msd} and to create detailed histograms for both the aged and the ordinary case. Here I find significant qualitative differences in the \gls{PDF} for different values of the parameters $\gamma$, $\nu$ and $\eta$, which gives insight into the convergence condition stated above. The results can mostly be explained heuristically by analyzing the influence of long steps on the process. I also find significant aging effects on the \gls{PDF}, which can be explained by a similar analysis.

An obvious goal for future work would be finding an analytical expression that can describe the features of the PDF quantitatively. One approach could be the incorporation of higher order terms into the asymptotic expansion in this work, which can be done straightforwardly. It is however not immediately clear how this would capture the effect of $\eta$ on the \gls{PDF}, so another approach might be more suitable.\\
It should also be noted that this thesis focuses very much on the theoretical aspects of the model, so it would be interesting to investigate how the different values of $\eta$ could be realized in experimental settings, with $\eta=\nu=2$ for a process with constant acceleration coming to mind.

In conclusion I find that the generalized L\'evy walk  does indeed allow for the recovery of the diffusion regimes that were divergent in the original model. The introduction of $\eta$ leaves the \gls{msd} unchanged but causes a variety effects in the \gls{PDF} that should be investigated further.

