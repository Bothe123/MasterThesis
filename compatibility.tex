\usepackage{ifpdf}

\ifpdf
%%das kann man benutzen, wenn man andere Formate benutzen will
%\DeclareGraphicsExtensions{{.pdf}}   %Endung der Grafiken, wenn nicht pdf
% die folgenden Angaben sind im PDF unter Datei | Dokumenteigenschaften 
% in Acrobat / Acrobat Reader sichtbar
% Aendern Sie bitte die Daten, wo noetig!
\usepackage[%
	pdftitle={\dctitle},
	pdfauthor={\dcauthorsurname\ \dcauthorname},
	pdfsubject={\dcpdfsubject}, % optional
	pdfkeywords={\dckeydea, \dckeydeb, \dckeydec, \dckeyded},
	pdfpagemode=UseOutlines,
  colorlinks=true,					% bitte nicht �ndern!
	linkcolor=black,					% bitte nicht �ndern!
	filecolor=black,					% bitte nicht �ndern!
	urlcolor=black,						% bitte nicht �ndern!
	citecolor=black,					% bitte nicht �ndern!
	pdftex=true,              % bitte nicht �ndern!
	plainpages=false,         % bitte nicht �ndern!
	hypertexnames=false,      % bitte nicht �ndern!
	pdfpagelabels=true,       % bitte nicht �ndern!
	hyperindex=true]{hyperref}% bitte nicht �ndern!
\else
  % hier kann mann eventuelle Befehle umdefinieren
  % die nur f�r pdfLaTeX vorgesehen sind
  % und das richtige Kompilieren durch den normalen LaTeX verhindern
	\newcommand{\texorpdfstring}[1]{#1}
\fi